\section{Ergebnisse}\label{sec:ergebnisse}
Die ankommenden Nachrichten erwarteten nicht nur eine Antwort, sondern auch eine Zustandsänderung des Systems. In der Dokumentation war auch genau beschrieben welche Aktion zu welcher Nachricht ausgeführt werden soll. So gibt es beispielsweise Nachrichten zum Anmelden eines neuen Interviewers oder zum Abfragen seines aktuellen Status. \\

Dies erforderte eine geeignete Datenstruktur um das System gut verwalten zu können. Hierfür wurden mehrere Klassen angelegt, um die nötigen Daten und Status der Interviewer aber auch der Projekte zu speichern, abzufragen und deren Zustände zu verändern. In der Dokumentation werden die Projekte als Gruppen bezeichnet. Diese Bezeichnung wird hier für die weitere Beschreibung übernommen. So sieht die Dokumentation es vor, dass die Interviewer verschiedene Zustände einnehmen können: Available, Busy und On-Hook. Von den Gruppen, welche eine Containerklasse für die Interviewer ist, können ebenfalls Zustände abfragt werden, zum Beispiel wie viele Interviewer momentan verfügbar, also nicht in einem aktiven Interview sind. \\

Die Anbindung an die Telefonanlage wurde aus Gründen des Zeitmanagements von meinem Betreuer übernommen. Er hatte bereits in einem vorherigen Versuch die Verbinndung zur Telefonanlage hergestellt und musste die vorliegende Anbindung nur leicht modifizieren, sodass diese auch mit der aktuellen Version des Servers funktioniert. \\

Zudem wurde eine eigene Logging Klasse implementiert. Diese kriegt im Konstruktor den Namen der Gruppe und legt eine Datei mit dem Gruppennamen im Dateiformat .log an. Die Idee hinter der Logging Klasse war den Code in logisch voneinander unabhängige Teile zu unterteilen. So waren im Code sehr viele print Statements vorhanden um den genauen Verlauf nachvollziehen zu können. Mit der Implementierung der Logging Klasse wurden die print Statements dorthin ausgelagert, sodass alle wichtigen Informationen nur an einer Stelle verarbeitet werden. Die Log Klasse schreibt bei jeder Fehlermeldung einen Eintrag in das Logfile um die Fehlersuche zu vereinfachen. Optional können auch alle Informationen in das Logfile geschrieben werden. Die Option des Loggings lässt sich über einen Übergabeparameter beim Aufruf in der Kommandozeile steuern. \\

Nachdem die implementierten Klassen und Funktionen implementiert und lokal getestet worden sind, wurde der Code ein weiteres Mal analysiert, um Feinheiten vorzunehmen und die Funktionen und Klassen zu kommentieren. \\

Das Ziel wurde also erreicht. Die Anbindung an den vorhanden Predicte Dialer wurde implementiert und in einem kleineren Ramen getestet. Die Inbetriebnahme wurde aus Zeitlichen Gründen noch nicht durchgeführt, jedoch ist es vorgesehen im Anschluss an das Praktikum als Werkstudent weiter an dem Projekt zu arbeiten. 
