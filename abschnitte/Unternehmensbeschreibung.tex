\section{Unternehmensbeschreibung}\label{sec:beschreibung}

\subsection{Unternehmensdaten}
„Umfragezentrum Bonn – Prof. Rudinger GmbH (uzbonn GmbH) - Gesellschaft für empirische Sozialforschung und Evaluation“ \newline
Gesellschaft mit beschränkter Haftung \newline
Im Folgenden abgekürzt als uzbonn bezeichnet. \newline

\textbf{Hauptsitz:} Oxfordstraße 15, 53111 Bonn \newline

\textbf{Standorte:} \newline
Bonn: \newline
Umfragezentrum Bonn - Prof. Rudinger GmbH, Oxfordstraße 15, 53111 Bonn \newline 
Umfragezentrum Bonn - Prof. Rudinger GmbH, Theaterstraße 18, 53111 Bonn \newline

\textbf{Gründungsjahr: } 2011 \newline

\textbf{Branchenzugehörigkeit: }  Markt- und Sozialforschung \newline

\textbf{Unternehmensleitung: } Prof. Dr. Georg Rudinger, Dr. Thomas Krüger und Dipl.-Kfm. Claus Mayerböck \newline


\textbf{Anzahl an Beschäftigten: } 123 \newline

\textbf{Unternehmensumsatz: } 123 \newline


\subsection{Angebotene Dienstleistungen }
uzbonn ist eine Ausgründung aus dem Zentrum für Evaluation und Methoden (ZEM) der Universität Bonn und bietet seit 2011 Lösungen für Fragestellungen und Forschungsinteressen in den Bereichen empirische Sozialforschung und Evaluation an. \\
Das Umfragezentrun hat langjährige Erfahrung in der sozialwissenschaftlichen Forschung, insbesondere in den Bereichen der Wirtschaft und Politik. Durch die enge Bindung an die universitäre Forschung verfügen die Mitarbeiter über eine ausgezeichnete Methodenkompetenz. Dadurch sind sie in der Lage, auf Bedürfnisse der Auftraggeber flexibel und passgenau zu reagieren, auch bei hochkomplexen Aufgabenstellungen. Außerdem fließen somit stets die neuesten wissenschaftlichen Erkenntnisse in die Arbeit ein. Das Unternehmen uzbonn bietet insgesamt ein ausgeglichenes Verhältnis zwischen Zuverlässigkeit, langjähriger Erfahrung und Innovation im Studiendesign.

\subsection{Organisationsstruktur}
abcdef

\subsection{Relevante Kunden und Projektbeispiele }
Die Kunden des Unternehmens uzbonn gliedern sich in drei Segmente: Institutionen, Fachhochschulen und Universitäten und Unternehmen. Die vorgestellten Kunden und die zugehörigen Projektbeispiele veranschaulichen die breite Ausrichtung und die diversen benötigten Fähigkeiten und Kompetenzen, die im Zuge der Beschäftigung im Unternehmen uzbonn relevant werden. \\

\subsubsection{Institutionen }
In erster Linie werden öffentlich-rechtliche und gemeinnützige Institutionen bedient. Zu diesen gehören unter anderem das Max-Planck-Institut für ausländisches und internationales Strafrecht oder das Bundesministerium für Arbeit und Soziales (BMAS). Im Zuge einer Befragung zur Arbeitszeit im öffentlichen Dienst im Auftrag der Bundestarifkommission von ver.di wurden beispielsweise 200.000 Beschäftigte in Einzel- und Gruppeninterviews befragt. Neben den Online-Befragungen, wurden die Papier-Fragebögen, die für bestimmte Arbeitsbereiche notwendig waren, durch uzbonn digital erfasst.\\

\subsubsection{Fachhochschulen und Universitäten }
Darüber hinaus zählen überwiegend deutsche Fachhochschulen und Universitäten, aber auch das Schweizerisches Institut für Entrepreneurship SIFE oder das Institute for Environmental Studies (IVM) der Universität Amsterdam zu den Kunden. In Zusammenarbeit mit den Universitäten Duisburg-Essen und Regensburg wurde das Projekt “ProSALAMANDER” durchgeführt. Dieses ermöglicht die Nachqualifizierung von zugewanderten Akademiker/-innen, um die Anerkennung von im Ausland erworbener Abschlüsse zu ermöglichen. Um projektbegleitend eine Evaluation der Umsetzung und nach Projektabschluss eine Untersuchung der Zielerreichung zu gewährleisten, wurde uzbonn über die Laufzeit des Projektes unterstützend eingesetzt. \\

\subsubsection{Unternehmen }
Auch verschiedene Unternehmen beispielsweise aus der Unternehmensberatung wie Roland Berger oder aus dem Bankwesen wie die Investitionsbank Berlin (IBB) nehmen die Leistungen des Unternehmens in Anspruch. Um eine Mitarbeiterbefragung durchzuführen, beauftragte die TÜV Rheinland Personal GmbH uzbonn mit der Befragung rund 3.500 Mitarbeiter/-innen. Der Fragebogen konnte hierbei sowohl online, als auch in Papierform ausgefüllt werden. Um der Vielfalt der Mitarbeiter/-innen gerecht zu werden, war der Fragebogen in zwölf verschiedenen Sprachen verfügbar und berücksichtigte dabei kulturelle Unterschiede, bspw. Besonderheiten des Umgangs und der Höflichkeit. \\





