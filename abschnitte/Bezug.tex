\section{Bezug zum bisherigen Studium }\label{sec:bezug}
Vor dem Studium an der Hochschule Bonn-Rhein-Sieg hatte ich noch keinerlei Erfahrung im Programmieren. Daher waren besonders die Veranstaltungen der ersten beiden Semester „Einführung in die Programmierung“ und „Programmierung 2“ sehr ausschlaggebend für den Erfolg des Praktikumsziels. Mit der Programmiersprache „Python“ war der erste Kontaktpunkt im dritten Semester in der Veranstaltung „Grundlagen von Wahrscheinlichkeitstheorie und Statistik”. Hier wurde eine gute Basis und ein gutes Verständnis für die Sprache entwickelt. Vertieft wurden die Kenntnisse in einer englischsprachigen Wahlpflicht Veranstaltung im vierten Semester „Scientific Programming with Python“, wo genauer auf die Sprache und die bestmögliche Anwendung eingegangen wurde. Das dritt Semester Modul „Algorithmen und Datenstrukturen und Graphentheorie“ hat zudem ein sehr gutes Verständnis für Algorithmen und Datenstrukturen aber auch für die Laufzeitabschätzung eines Programms vermittelt.